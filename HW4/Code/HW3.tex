\ProvidesPackage{commands}
\documentclass[11pt]{article}
\usepackage{epstopdf}
\usepackage{subfigure,graphicx}
\usepackage{amsmath}
\usepackage{epsf}
\usepackage{amsfonts}
\usepackage{amssymb}
\usepackage{color}
\usepackage{mathtools}
\usepackage{placeins}
\usepackage{booktabs}
\usepackage{enumitem}
\usepackage{caption}
\usepackage[margin=0.8in, paperwidth=8.5in, paperheight=11in]{geometry}
\usepackage{amsfonts}
\usepackage{amsmath}
\usepackage{amsbsy}
\usepackage{authblk}
\usepackage{graphicx}
\usepackage{listings}
\usepackage{array}
\usepackage{titlesec}
\usepackage{amssymb}
\usepackage{bm}
\usepackage{mathtools}
\usepackage{titlesec}

\usepackage[latin1]{inputenc}\newcommand{\bs}[1]{\boldsymbol{#1}}
\newcommand{\del}[2]{\frac{\partial {#1}}{\partial {#2}}}
\newcommand{\D}[2]{\frac{D^{\overline{\alpha}}}{\overline{\alpha !}}{#1}(#2,#2)\ {\bf x}^{\overline{\alpha}}}
\newcommand{\dv}[3]{\frac{{\rm d}^{#1}{#2}}{d{#3}^{#1}}}
\newcommand{\ddel}[5]{\frac{\partial^{ {#1} + {#2}} {#3}}{\partial {#4}^{#1} \partial{#5}^{#2}}}
\newcommand{\dev}{{\rm {\bf dev}}}
\newcommand{\proj}[1]{\frac{1}{R^2}{\bf X}\otimes{\bf X}}
\newcommand{\Ie}[1]{I^{\rm e}_{#1}}
\newcommand{\Ce}[1]{\bf C^{\rm e^{#1}}}
\newcommand{\Fe}[2]{F^{\rm e^{#2}}_{#1}}
\newcommand{\Fv}[2]{F^{\rm v^{#2}}_{#1}}
\newcommand{\f}[2]{f^{\rm {#2}}_{#1}}
\newcommand{\B}[2]{B^{\rm {#2}}_{#1}}
\newcommand{\E}[2]{E^{\rm {#2}}_{#1}}
\newcommand{\fv}[2]{f^{\rm v^{#2}}_{#1}}
\newcommand{\dfv}[2]{\dot{f}^{\rm v^{#2}}_{#1}}
\newcommand{\tGam}[2]{\tilde{\Gamma}^{\rm v^{#2}}_{#1}}
\newcommand{\Gam}[2]{\Gamma^{\rm v^{#2}}_{#1}}
\newcommand{\A}[1]{\mathcal{A}_{#1}}
\newcommand{\F}[2]{F^{\rm #2}_{#1}}
\newcommand{\hpeq}{\hat{\psi}^{\rm Eq}}
\newcommand{\hpneq}{\hat{\psi}^{\rm NEq}}
\newcommand{\etak}{\eta_K({I_1,I_2,J},{\bf C^{\rm e}, B^{\rm v}})}
\newcommand{\nuk}{\nu_K({I_1,I_2,J},{\bf C^{\rm e}, B^{\rm v}})}
\newcommand{\thetak}{\theta_K({I_1,I_2,J},{\bf C^{\rm e}, B^{\rm v}})}
\newcommand{\etaj}{\eta_J({I_1,I_2,J},{\bf C^{\rm e}, B^{\rm v}})}
\newcommand{\dFv}[2]{\dot{F}^{\rm v^{#2}}_{#1}}
\newcommand{\hatpsi}{\widehat{\psi}(I_1, I_2,I^{\rm e}_1,I^{\rm e}_2,J)}
\newcommand{\hpsi}{\widehat{\psi}(I_1,I^{\rm e}_1,J)}
\newcommand{\Fh}[1]{\widehat{\mathcal{F}}\left({\bf F, \Fv{}{}}, {#1}\right)}
\newcommand{\Fhstar}[1]{\widehat{\mathcal{F}}^*\left({\bf F, \Fv{}{}}, {#1}\right)}
\newcommand{\sbar}{\overline{\bm{\sigma}}}
\newcommand{\hpsicomp}[1]{\sum_{r=1}^{2}\left\{\frac{3^{1-\alpha_r}}{2\alpha_r}\mu_r(I^{\alpha_r}_1-3^{\alpha_r})
+\frac{3^{1-a_r}}{2a_r}m_r({\Ie{1}}^{^{a_r}}-3^{a_r})\right\}
+\mu{#1}+\kappa{#1}^2}
\newcommand{\Ni}[1]{N^{(e)}_i(#1)}
\newcommand{\hNi}[1]{\hat{{N}}^{(e)}_i(#1)}
\newcommand{\Ld}{L^{\dagger}}
\newcommand{\intinfinf}{\int_{-\infty}^{\infty} \int_{-\infty}^{\infty}}
\newcommand{\LLnorm}[1]{\left\lVert{#1}\right\rVert_2}
\newcommand{\Linorm}[1]{{\left\lVert{#1}\right\rVert_\infty}}
\newcommand{\tr}{\rm tr}
\newcommand{\deldel}[2]{\frac{\partial^2 {#1}}{\partial {#2}^2}}
\newcommand{\kd}[1]{\delta_{#1}}
\newcommand{\Fie}[1]{{\bf F}^{#1}}
\newcommand{\Comp}{\emph{CompStrainStress\_Cee570.m}}
\newcommand{\Comps}{\emph{CompStrainStress\_Elem\_Cee570.m}}
\newcommand{\Feap}{\emph{FEA\_Program.m}}
\newcommand{\Elast}{\emph{Elast2d\_Elem.m}}
\newcommand{\Assem}{\em{AssemStifForc.m}}
\newcommand{\Fb}{\em{F\_bar\_int}}
\newcommand{\Form}{\em{FormFE.m}}
\newcommand{\Sol}{\em{SolveFE.m}}
\newcommand{\inpt}{\em{triangtwo.m}}

\titlespacing\section{10pt}{10pt plus 4pt minus 2pt}{10pt plus 2pt minus 2pt}
\titlespacing\subsection{0pt}{8pt plus 4pt minus 2pt}{8pt plus 2pt minus 2pt}
\titlespacing\subsubsection{0pt}{12pt plus 4pt minus 2pt}{6pt plus 2pt minus 2pt}
\titlespacing*{\title}{-2ex}{*-2ex}{-2ex}
\usepackage{color} %red, green, blue, yellow, cyan, magenta, black, white
\definecolor{mygreen}{RGB}{28,172,0} % color values Red, Green, Blue
\definecolor{mylilas}{RGB}{170,55,241}
\setlength\parindent{0pt}
\graphicspath{{Figures/}}

\title{\bf CEE 576: Nonlinear Finite Elements \\ HW Assignment 4}
\author{Bhavesh Shrimali \\ NetID: bshrima2}
\date{\today}
\begin{document}
\maketitle \hrule \hrule \hrule
\section*{Sol$^n$ 1: }
Given Deformation Mapping: 
\[
\bm{\phi}({\bf z})
=
z_1(1+z_2){\bf e}_1
+ z_2(1+3z_1){\bf e}_2
+ z_3{\bf e}_3
\]
The deformation gradient can be computed as follows:
\begin{align*}
{\bf F}
=
\del{\bm\phi}{\bf z}({\bf z})=
\begin{bmatrix}
1+z_2 & z_1 & 0\\
3z_2 & 1+3z_1 & 0\\
0 & 0 & 1
\end{bmatrix}
\end{align*}
The Lagrangian Strain Tensor is given by
\begin{align*}
{\bf E}
& =
\frac{\bf \F{}{T}\F{}{}-I}{2} \\
& = \frac{1}{2}
\begin{bmatrix}
(1+z_2)^2 + 9z^2_2 & 3z_2(1+3z_1)+z_1(1+z_2) & 0\\
9z^2_2 & 3z_2(1+3z_1)+z_1(1+z_2) & (1+3z_1)^2 + z_1^2 & 0\\
0 & 0 & 1
\end{bmatrix}-\frac{1}{2}
\begin{bmatrix}
1 & 0 & 0 \\
0 & 1 & 0\\
0 & 0 & 1
\end{bmatrix}\\
& = 
\frac{1}{2}
\begin{bmatrix}
(1+z_2)^2 + 9z^2_2 -1 & 3z_2(1+3z_1)+z_1(1+z_2) & 0\\
9z^2_2 & 3z_2(1+3z_1)+z_1(1+z_2) & (1+3z_1)^2 + z_1^2 - 1 & 0\\
0 & 0 & 0
\end{bmatrix}
\end{align*}
The Green deformation strain tensor is given by
\begin{align*}
{\bf C}
& = {\bf \F{}{T} F}\\
& = \begin{bmatrix}
(1+z_2)^2 + 9z^2_2 & 3z_2(1+3z_1)+z_1(1+z_2) & 0\\
9z^2_2 & 3z_2(1+3z_1)+z_1(1+z_2) & (1+3z_1)^2 + z_1^2 & 0\\
0 & 0 & 1
\end{bmatrix}
\end{align*}
Substituting the coordinates respectively we get
\begin{align*}
{\bf E}
& =
\begin{bmatrix}
6 & 7 & 0\\
7 & 8 & 0\\
0 & 0 & 0
\end{bmatrix}\\
{\bf C}
& =
\begin{bmatrix}
13 & 14 & 0\\
14 & 17 & 0\\
0 & 0 & 1
\end{bmatrix}
\end{align*}
The principal stretches are the eigenvalues, and the principal directions are the corresponding eigenvectors, of $\bf C$. Therefore
\begin{align*}
\det\left({\bf C}-\lambda{\bf I}\right) = 0
\implies 
\begin{vmatrix}
13-\lambda & 14 & 0\\
14 & 17-\lambda & 0\\
0 & 0 & 1-\lambda
\end{vmatrix} = 0 \\ 
\lambda = 1\ \ \ \ \text{or}\ \ \ \lambda^2 - 30\lambda + 25 = 0 \\
\lambda = 1,15\pm10\sqrt{2}
\end{align*}
Hence the principal stretches are $0,15\pm10\sqrt{2}$
Now the corresponding principal directions, which are the corresponding eigenvectors, as follows: 
\begin{itemize}
\item Corresponding to $\lambda = 1$
\begin{align*}
{\bf v}^{(1)}
=
\begin{Bmatrix}
0\\0\\1
\end{Bmatrix}
\end{align*}
\item Corresponding to $\lambda=15-10\sqrt{2}$
\begin{align*}
{\bf v}^{(2)}
=
\begin{Bmatrix}
-0.755\\0.655\\0
\end{Bmatrix}
\end{align*}
\item Corresponding to $\lambda = 15+10\sqrt{2}$
\begin{align*}
{\bf v}^{(3)}
=
\begin{Bmatrix}
0.655\\-0.755\\0
\end{Bmatrix}
\end{align*}
\end{itemize}\hrule
\section*{Sol$^n$ 2:}
Given deformation mapping
\begin{align*}
u({\bf X},t)
=
{\bf N}({\bf X}){\bf u}(t)
=
\frac{1}{l_0}
[X_2 - X\ \ X-X_1]
\begin{Bmatrix}
u_1(t)\\u_2(t)
\end{Bmatrix}
\end{align*}
where $l_0 = X_2 - X_1$
\section*{Sol$^n$ 3: }
Given motion 
\[
x = X + Yt\ ; \ \ \ y = Y + \frac{1}{2}Xt
\]
\subsection*{(a)}
The position coordinates of the element as $t=1$ can be sketched as follows: \\
The Deformation gradient, $\bf F$, associated with the given deformation mapping, $\bm\chi({\bf X})$, can be given by:
\begin{align*}
{\bf F} ({\bf X},t)
& = \del{\bm\chi}{\bf X}({\bf X}, t)\\
& = \begin{bmatrix}
1 & t\\
t/2 & 1
\end{bmatrix}
\end{align*}
The green strain tensor can be calculated as follows:
\begin{align*}
{\bf C} 
& = {\bf \F{}{T} F} = \begin{bmatrix}
1+t^2/4 & 3t/2\\
3t/2 & 1+t^2
\end{bmatrix}\\
{\bf E}
& =
\frac{1}{2}
({\bf \F{}{T} F - I})
=
\frac{1}{2}
\begin{bmatrix}
t^2/4 & 3t/2\\
3t/2 & t^2
\end{bmatrix}
=
\frac{t}{8}
\begin{bmatrix}
t & 6\\
6 & 4t
\end{bmatrix}
\end{align*}
At $t=1$, 
\begin{align*}
{\bf C}
=
\frac{1}{4}
\begin{bmatrix}
5 & 6\\
6 & 8
\end{bmatrix}\ ;\ \ \ \ \ {\bf E} = \frac{1}{8}\begin{bmatrix}
1 & 6\\
6 & 4
\end{bmatrix}
\end{align*}
\subsection*{(b):}
The displacement field can be given as: 
\[
{\bf u} = {\bm\chi}({\bf X},t) - {\bf X}
\]
The velocity and accelerations can be calculated as follows: 
\begin{align*}
{\bf v}({\bf X},t)
& = \del{{\bf u}}{t}({\bf X},t) 
= 
\del{\bm\chi}{t}({\bf X},t) \\
& = 
\del{(X+Yt)}{t}{\bf e}_1 
+ 
\del{(Y+\frac{1}{2}Xt)}{t}{\bf e}_2 \\
{\bf v}({\bf X},1)&  = 
Y{\bf e}_1
+ 
\frac{1}{2}X{\bf e}_2 \\
{\bf a}({\bf X},t)
& =
\deldel{{\bf u}}{t}({\bf X},t) = {\bf 0}
\end{align*}
\subsection*{(c):}
The rate of deformation is given by
\begin{align*}
L_{ij}
& =
\del{v_i}{x_j} = \del{v_i}{X_k}\del{X_k}{x_j} = \del{v_i}{X_k}\F{kj}{-1}
\end{align*} 
In abstract notation this can be written as
\begin{align*}
{\bf L} & = \del{\bf v}{\bf X}{\bf \F{}{-1}}\\ \\
\del{\bf v}{\bf X} & = \begin{bmatrix}
0 & 1\\
1/2 & 0 
\end{bmatrix}\ ; \ \ \ \ \ {\bf \F{}{-1}} = \frac{1}{t^2-2} \begin{bmatrix}
-2 & 2t \\
t & -2
\end{bmatrix}
\end{align*}
Thus we finally get the deformation rate tensor as 
\begin{align*}
{\bf L}
=
\frac{1}{t^2-2}
\begin{bmatrix}
t & -2\\
-1 & t
\end{bmatrix}
\end{align*}
$\bf L$ can be decomposed into its symmetric and skew-symmetric parts which are the rate-of-deformation ($\bf D$) and the spin tensor ($\bm\Omega$) respectively 
\begin{align*}
{\bf D} 
& = \frac{1}{2}({\bf L}+{\bf L}^T)
= \frac{1}{t^2-2}
\begin{bmatrix}
t & -3/2\\
-3/2 & t
\end{bmatrix}\\ \\
{\bm\Omega}
& = \frac{1}{2}({\bf L}-{\bf L}^T)
=
\frac{1}{t^2-2}
\begin{bmatrix}
0 & -1/2\\
1/2 & 0
\end{bmatrix}
\end{align*}
The corresponding values at $t=1$ are 
\begin{align*}
{\bf D}
& = 
\begin{bmatrix}
-1.0 & 1.5 \\
1.5 & -1.0
\end{bmatrix} \\
{\bm\Omega}
& = 
\begin{bmatrix}
0 & 0.5 \\
-0.5 & 0
\end{bmatrix}
\end{align*}
\subsection*{(d):}
The above tensors, when evaluated at $t=0.5$, respectively are:
\begin{align*}
{\bf D}
& = 
\begin{bmatrix}
-0.2857 & 0.8571\\
0.8571 & -0.2857
\end{bmatrix} \\ \\
{\bm\Omega}
& =
\begin{bmatrix}
0 & 0.2857\\
-0.2857 & 0
\end{bmatrix}
\end{align*}
\subsection*{(e):}
The determinant of the deformation gradient, $\bf F$, given by $J=\det\bf F$ is
\[
J = \det \begin{bmatrix}
1 & t\\
0.5t & 1
\end{bmatrix} = 1-0.5t^2
\]
We know that for any physically realizable deformation mapping $\bm\chi$, we should have 
\begin{align*}
J>0\ \ \  \forall t > 0 \implies 1-0.5t^2 > 0\ \ \ \ \ \text{or}\ \ \ \ 0<t<\sqrt{2}
\end{align*}
Hence the Jacobian of transformation remains positive for $t\in [0,\sqrt{2})$ and changes sign at $t=\sqrt{2}$. The Jacobian vanishes at $t=\sqrt{2}$, hence violating the conservation of mass. The deformation mapping is not physically realizable beyond that point.  
\end{document}